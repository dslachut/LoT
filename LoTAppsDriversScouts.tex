%This work is licensed under the Creative Commons Attribution-ShareAlike 4.0 
%International License. To view a copy of this license, visit 
%http://creativecommons.org/licenses/by-sa/4.0/deed.en_US.

\documentclass[11pt]{beamer}
%\usetheme{default}
%\usetheme{Boadilla}
%\usetheme{Madrid}
%\usetheme{Pittsburgh}
%\usetheme[height=7mm]{Rochester}
%\usetheme{Copenhagen}
%\usetheme{Warsaw}
\usetheme{Singapore}
%\usetheme{Malmoe}
%\usetheme{Cambridge}
\usepackage[utf8]{inputenc}
\usepackage{amsmath}
\usepackage{amsfonts}
\usepackage{amssymb}
\usepackage{hyperref}
\author{David Lachut}
\title{Writing Apps, Drivers, and Scouts for the LoT Hub}
%\setbeamercovered{transparent} 
%\setbeamertemplate{navigation symbols}{} 
%\logo{} 
%\institute{} 
%\date{} 
%\subject{} 
\begin{document}

\begin{frame}
\titlepage
\end{frame}

\begin{frame}{Agenda}
\tableofcontents
\end{frame}

\section{First Things}

\begin{frame}{Dude, you're not Dr Banerjee ...}
    \begin{itemize}
        \item My name is David
        \item PhD student since 2011
        \item Working on HomeOS/LoT Hub since pre-Alpha
        \item dlachut1@umbc.edu
    \end{itemize}
\end{frame}

\begin{frame}{Terms}
    \begin{itemize}
        \item[Port]     An access point for services
        \item[Role]     An agreed upon description of what a service can do
        \item[Module]   A software addin to Hub that accesses Ports
        \item[App]      A module that consumes services from Ports
        \item[Driver]   A module that provides services via Ports
        \item[Scout]    A helper program to scan for hardware and activate a driver
        \item[Platform] The control layer between Apps, Drivers, and Scouts
    \end{itemize}
\end{frame}

\begin{frame}{}
    \begin{itemize}
        \item Does everyone have their Dev environment set-up with dependencies installed?
    \end{itemize}
\end{frame}

\begin{frame}{The rest of the lecture}
    \begin{itemize}
        \item 1h 15m is not enough time to thoroughly cover all the necessary details
        \item Even if it were, you probably wouldn't learn it without working along with me
        \item This talk will be an overview to make it easier when you read the docs
    \end{itemize}
\end{frame}

\section{Make an App}

\begin{frame}{Agenda}
\tableofcontents
\end{frame}

\begin{frame}{Overview}
    \begin{itemize}
        \item Create a VS Project
        \item Set Names, Properties, and References
        \item Implement the Abstract Methods
    \end{itemize}
\end{frame}

\begin{frame}{Create a VS Project}
    \begin{itemize}
        \item In Apps section of the solution
        \item Add a C\# Class Library
        \item Place in ``Apps'' directory
        \item Override the Abstract Methods
    \end{itemize}
\end{frame}

\begin{frame}{Set Names, Properties, and References}
    \begin{itemize}
        \item Source: Rename the class
        \item Source: Rename the Namespace
        \item Source: Add `using' declarations
        \item Source: Add property declaration
        \item Source: Inherit from ModuleBase
        \item Source: Generate Stubs for Abstract Methods
    \end{itemize}
\end{frame}

\begin{frame}{Set Names, Properties, and References}
    \begin{itemize}
        \item References: Add `Common', `Views', and `DataStore'
        \item References: Add `AddIn', `ServiceModel', and `ServiceModel.Web'
    \end{itemize}
\end{frame}

\begin{frame}{Set Names, Properties, and References}
    \begin{itemize}
        \item Properties: Set the output path
        \item Properties: Set the Assembly Name
        \item Properties: Set the Default namespace
    \end{itemize}
\end{frame}

\begin{frame}{Override the Abstract Methods}
    \begin{itemize}
        \item \texttt{Start()} Entry point for the App
        \item \texttt{Stop()}  Clean up after the App
        \item \texttt{PortRegistered(VPort port)}\\ Called when a port is instantiated somewhere on the hub
        \item \texttt{PortDeregistered(VPort port)}\\ Called when a port is removed from the hub
        \item \texttt{OnNotification(string Rolename, string OpName, IList<VParamType> retVals, VPort senderPort)}\\ Called when a notification is issued from a subscribed port
    \end{itemize}
\end{frame}

\section{Make a Driver}

\begin{frame}{Agenda}
\tableofcontents
\end{frame}

\begin{frame}{Overview}
    \begin{itemize}
        \item Create a VS Project
        \item Set Names, Properties, and References
        \item Implement the Abstract Methods
    \end{itemize}
\end{frame}

\begin{frame}{Create a VS Project}
    \begin{itemize}
        \item In Drivers section of the solution
        \item Add a C\# Class Library
        \item Place in ``Drivers'' directory
        \item Override the Abstract Methods
    \end{itemize}
\end{frame}

\begin{frame}{Set Names, Properties, and References}
    \begin{itemize}
        \item Source: Rename the class
        \item Source: Rename the Namespace
        \item Source: Add `using' declarations
        \item Source: Add property declaration
        \item Source: Inherit from ModuleBase
        \item Source: Implement abstract Methods
        \item References: Add `Common', `Views', and `DataStore'
        \item References: Add `AddIn', `ServiceModel', and `ServiceModel.Web'
        \item Properties: Set the output path
        \item Properties: Set the Assembly Name
        \item Properties: Set the Default namespace
    \end{itemize}
\end{frame}

\begin{frame}{Override the Abstract Methods}
    \begin{itemize}
        \item \texttt{Start()} Entry point for the App
        \item \texttt{Stop()}  Clean up after the App
        \item \texttt{PortRegistered(VPort port)}\\ Called when a port is instantiated somewhere on the hub
        \item \texttt{PortDeregistered(VPort port)}\\ Called when a port is removed from the hub
        \item \texttt{OnNotification(string Rolename, string OpName, IList<VParamType> retVals, VPort senderPort)}\\ Called when a notification is issued from a subscribed port
    \end{itemize}
\end{frame}

\begin{frame}{Override the Abstract Methods}
    \begin{itemize}
        \item \texttt{PortDeregistered(VPort port)}
        \item \texttt{PortRegistered(VPort port)}
        \item \texttt{OnNotification(string Rolename, string OpName, IList<VParamType> retVals, VPort senderPort)}
        \item A driver can usually have these methods do nothing, because it usually doesn't care about other ports
    \end{itemize}
\end{frame}



\section{Make a Scout}

\begin{frame}{Agenda}
\tableofcontents
\end{frame}

\begin{frame}{Overview}
    \begin{itemize}
        \item Create a VS Project
        \item Set Names, Properties, and References
        \item Implement the Abstract Methods
    \end{itemize}
\end{frame}

\begin{frame}{Create a VS Project}
    \begin{itemize}
        \item In Scouts section of the solution
        \item Add a C\# Class Library
        \item Place in ``Scouts'' directory
        \item Override the Abstract Methods
    \end{itemize}
\end{frame}

\begin{frame}{Set Names, Properties, and References}
    \begin{itemize}
        \item Source: Rename the class
        \item Source: Rename the Namespace
        \item Source: Add `using' declarations
        \item Source: Don't need property declaration
        \item Source: Inherit from IScout
        \item Source: Implement abstract Methods
        \item References: Add `Common', `Views', and `DeviceScout'
        \item References: Add `AddIn', `ServiceModel', and `ServiceModel.Web'
        \item Properties: Set the output path
        \item Properties: Set the Assembly Name
        \item Properties: Set the Default namespace
    \end{itemize}
\end{frame}

\begin{frame}{Set Names, Properties, and References}
    \begin{itemize}
        \item Easiest way to implement the Scout is to copy the DummyScout and change the names as appropriate
        \item Only substantive differences will be in the \texttt{GetDevices()} method
    \end{itemize}
\end{frame}

\section{RTFM}

\begin{frame}{Agenda}
\tableofcontents
\end{frame}

\begin{frame}{RTFM}
    \begin{itemize}
        \item I hope this has been a decent orientation.
        \item You'll probably need to \\
        
        \href{https://labofthings.codeplex.com/downloads/get/764896}{\large RTFM: Read The \textit{Fine} Manuals} at
        
        {\small\url{https://labofthings.codeplex.com/downloads/get/764896} [PDF WARNING]}
        
        \item Look here, too:
        
        {\small\url{http://labofthings.codeplex.com/documentation}}
        
    \end{itemize}
\end{frame}

\end{document}